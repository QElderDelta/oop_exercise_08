\documentclass[a4paper, 12pt]{article}
\usepackage{cmap}
\usepackage[12pt]{extsizes}			
\usepackage{mathtext} 				
\usepackage[T2A]{fontenc}			
\usepackage[utf8]{inputenc}			
\usepackage[english,russian]{babel}
\usepackage{setspace}
\singlespacing
\usepackage{amsmath,amsfonts,amssymb,amsthm,mathtools}
\usepackage{fancyhdr}
\usepackage{soulutf8}
\usepackage{euscript}
\usepackage{mathrsfs}
\usepackage{listings}
\pagestyle{fancy}
\usepackage{indentfirst}
\usepackage[top=10mm]{geometry}
\rhead{}
\lhead{}
\renewcommand{\headrulewidth}{0mm}
\usepackage{tocloft}
\renewcommand{\cftsecleader}{\cftdotfill{\cftdotsep}}
\usepackage[dvipsnames]{xcolor}

\lstdefinestyle{mystyle}{ 
	keywordstyle=\color{OliveGreen},
	numberstyle=\tiny\color{Gray},
	stringstyle=\color{BurntOrange},
	basicstyle=\ttfamily\footnotesize,
	breakatwhitespace=false,         
	breaklines=true,                 
	captionpos=b,                    
	keepspaces=true,                 
	numbers=left,                    
	numbersep=5pt,                  
	showspaces=false,                
	showstringspaces=false,
	showtabs=false,                  
	tabsize=2
}

\lstset{style=mystyle}

\begin{document}
\thispagestyle{empty}	
\begin{center}
	Московский авиационный институт
	
	(Национальный исследовательский университет)
	
	Факультет "Информационные технологии и прикладная математика"
	
\end{center}
\vspace{40ex}
\begin{center}
	\textbf{\large{Лабораторная работа №7 по курсу \textquotedblleft Объектно-ориентированное программирование\textquotedblright}}
\end{center}
\vspace{40ex}
\begin{flushright}
	\textit{Студент: } Живалев Е.А.
	
	\vspace{2ex}
	\textit{Группа: } М8О-206Б
	
	\vspace{2ex}
	\textit{Преподаватель: } Журавлев А.А.
	
	\vspace{2ex}
	\textit{Вариант: } 5
	
	\vspace{2ex}
	\textit{Оценка: } \underline{\quad\quad\quad\quad\quad\quad}
	
	 \vspace{2ex}
	\textit{Дата: } \underline{\quad\quad\quad\quad\quad\quad}
	
\end{flushright}

\begin{vfill}
	\begin{center}
		Москва
		
		2019
	\end{center}	
\end{vfill}
\newpage
\section{Исходный код}

Ссылка на github : https://github.com/QElderDelta/oop\_exercise\_07

\vspace{3ex}
\textbf{\large{figure.hpp}}
\lstinputlisting[language=C++]{figure.hpp}

\vspace{3ex}
\textbf{\large{rhombus.hpp}}
\lstinputlisting[language=C++]{rhombus.hpp}

\vspace{3ex}
\textbf{\large{rhombus.cpp}}
\lstinputlisting[language=C++]{rhombus.cpp}

\vspace{3ex}
\textbf{\large{pentagon.hpp}}
\lstinputlisting[language=C++]{pentagon.hpp}

\vspace{3ex}
\textbf{\large{pentagon.cpp}}
\lstinputlisting[language=C++]{pentagon.cpp}

\vspace{3ex}
\textbf{\large{hexagon.hpp}}
\lstinputlisting[language=C++]{hexagon.hpp}

\vspace{3ex}
\textbf{\large{hexagon.cpp}}
\lstinputlisting[language=C++]{hexagon.cpp}

\vspace{3ex}
\textbf{\large{point.hpp}}
\lstinputlisting[language=C++]{point.hpp}

\vspace{3ex}
\textbf{\large{pubsub.hpp}}
\lstinputlisting[language=C++]{pubsub.hpp}

\vspace{3ex}
\textbf{\large{main.cpp}}
\lstinputlisting[language=C++]{main.cpp}


\vspace{3ex}
\textbf{\large{CMakeLists.txt}}
\lstinputlisting{CMakeLists.txt}
\newpage
\section{Тестирование}
\vspace{3ex}

\textbf{test\_1\_argv.txt}:

Создадим буфер размера один и добавим туда ромб, пятиугольник и шестиугольник.

1 - Rhombus, 2 - Pentagon, 3 - Hexagon

Rhombus: [0.000, 0.000] [0.000, 0.000] [0.000, 0.000] [0.000, 0.000] 

1 - Rhombus, 2 - Pentagon, 3 - Hexagon

Pentagon: [0.000, 0.000] [0.000, 0.000] [0.000, 0.000] [0.000, 0.000] [0.000, 0.000] 

1 - Rhombus, 2 - Pentagon, 3 - Hexagon

Hexagon:[0.000, 0.000] [0.000, 0.000] [0.000, 0.000] [0.000, 0.000] [0.000, 0.000] [0.000, 0.000] 

Полученные файлы:

qelderdelta@qelderdelta-UX331UA:\~{}/Study/oop\_exercise\_08/build\$ cat 1198.txt 

Rhombus: [0.000, 0.000] [0.000, 0.000] [0.000, 0.000] [0.000, 0.000] 

qelderdelta@qelderdelta-UX331UA:\~{}/Study/oop\_exercise\_08/build\$ cat 214.txt 

Pentagon: [0.000, 0.000] [0.000, 0.000] [0.000, 0.000] [0.000, 0.000] [0.000, 0.000] 

qelderdelta@qelderdelta-UX331UA:\~{}/Study/oop\_exercise\_08/build\$ cat 807.txt 

Hexagon:[0.000, 0.000] [0.000, 0.000] [0.000, 0.000] [0.000, 0.000] [0.000, 0.000] [0.000, 0.000] 
\section{Объяснение результатов работы программы}

При вводе координат для создания ромба производится проверка этих координат, ведь они могут не образовывать ромб. Для этого реализована функция checkIfRhombus, которая вычисляет расстояния от одной точки до трёх остальных, а поскольку фигура является ромбом, то два из низ должны быть равны. Третье же значение функция возвращает ведь оно равно длине одной из диагоналей. Площадь ромба вычисляется как половина произведения диагоналей, центр - точка пересечения диагоналей. Методы вычисления площади и центра для пяти- и шестиугольника совпадают. Чтобы найти площадь необходимо перебрать все ребра и сложить площади трапеций, ограниченных этими ребрами. Чтобы найти центр необходимо разбить фигуры на треугольники(найти одну точку внутри фигуры), для каждого треугольника найти центроид и площадь и перемножить их, просуммировать полученные величины и разделить на общую площадь фигуры.

\newpage
\section{Выводы}

В ходе выполнения работы я познакомился с тем, как устроены встроенные механизмы языка для разработки многопоточных программ, а также получил навыки их написания.
\end{document}